\documentclass{report}
\usepackage[latin1]{inputenc}
\usepackage{xspace}
%\usepackage[francais]{babel}
\usepackage[T1]{fontenc}
%\usepackage{algorithm}
%\usepackage{algorithmic}
%\usepackage{graphicx} 
\usepackage{bm}
\usepackage{subcaption}
\usepackage{amsmath}
\usepackage{amsfonts}

\title{Hand recognition on a video with active contour}
\author{Paul Schreiner, Antun Aleksa}
\date{28/11/2016}

\begin{document}
\maketitle



\section*{1. Two ideas for color based hand detection}

\subsection*{Backprojection detection}

If the hand is in the center of the image, we can select a small square at the center of the image which will be on this hand. Its histogram gives a description of the pixel colors of the hand. We then use the backprojection algorithm to detect the whole hand: we use the histogram computed with the small rectangle to record how well the pixels of the whole image fit the distribution in this histogram using this algorithm. This method is quite effective even in poor lighting conditions, but the hand on the first video image needs to be centered. 

\subsection*{YUV color space}

The skin can be easily detected using the YUV color space: Y represents intensity and U,V (Cb,Cr) represent chromatic distances to pure blue or pure red. Intensity is not very significant to detect skin, but we can find narrow intervals for U and V which represent skin color. However, this method might not work in poor lighting conditions. 



\section*{2. Active Contour detection}

Active contour model is used to detect an outline of a certain object in an image. This model is apropriate for our problem because it is used when the approximate shape of the boundary is known from before (in our case the shape of a hand) and it can be used to track dynamic objects. 

\section*{3. Algorithm}

\begin{enumerate}

\item \textbf{Hand detection} (Backprojection algorithm or YUV space color detection)\\- currently testing which one is better in terms of precision and speed

\item \textbf{Noise filtering} (not implemented yet)

\item \textbf{Initialization of the snake} (not implemented yet)

\item \textbf{Active contour detection} (parameters to optimize)

\end{enumerate}

\section*{4. Tools}

We chose python because we are familiar with the language and we have a lot of great libraried on our disposal. We decided to use OpenCV to detect colors and run the backprojection or YUV color detection on the image, and Scikit-Image because of it's built in function which can be use to find the active contour. \\

\textbf{Language:} Python\\

\textbf{Libraries:} OpenCV (Color detection), Scikit-Image (Active Contour)\\






\end{document}